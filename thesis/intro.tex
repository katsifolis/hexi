\section{Εισαγωγή}
\subsection{Το Πρόβλημα}
Τα προγράμματα επεξεργασίας δυαδικών αρχείων (τα λεγόμενα προγράμματα επεξεργασίας \emph{HEX editors}) είναι προγράμματα που προορίζονται για επεξεργασία αρχείων που δεν ερμηνεύονται παρά ως μπλοκ δεδομένων.
Οι hex editors επί της ουσίας μπορούν να επεξεργαστούν - διαβάσουν όλα τα είδη αρχείων είτε αποτελούν εκτελέσιμα είτε όχι. Για παράδειγμα αρχεία πολυμέσων όπως png, jpg, gif, mp4, mkv, mp3, ogg και εκτελέσιμα αρχεία όπως .exe των \emph{Windows}, elf των \emph{Linux}, apk των \emph{Android}.

% Basic reasons
Ενδέχεται να υπάρχουν αρκετοί λόγοι για την επεξεργασία τους από έναν \emph{hex editor} όπως η μορφή αρχείου να είναι άγνωστη, η κεφαλίδα αρχείου να είναι κατεστραμμένη συνεπώς το αρχείο να είναι αδύνατο να ανοιχτεί ή ακόμα και να υπάρχει εξειδικευμένο λογισμικό για τη δεδομένη μορφή.
% Recover deleted files
Ένα άλλο παράδειγμα είναι η ανάκτηση διαγραμμένων αρχείων από τον σκληρό δίσκο. Η τεχνική της εύρεσης των κομματιών-μπλοκ για την `συναρμολόγηση` ενός διεγραμμένου αρχείου ονομάζεται \emph{file carving}.
% Forensics
Επίσης τα byte στην αρχή και στο τέλος ενός αρχείου (κεφαλίδα - header) διατίθονται για συγκεκριμένες πληροφορίες και μεταδεδομένα.
Αυτό είναι σημαντικό για τους ανθρώπους στον κλάδο του \emph{digital forensic} επειδή κακόβουλοι χρήστες θα αλλάξουν την επέκταση (και συνεπώς την κεφαλίδα) ενός αρχείου για να `καμουφλάρουν` αυτό το αρχείο από την μια μορφή που είναι σε κάποια άλλη.

Κάποιες από τις πιο περίπλοκες λειτουργίες περιλαμβάνουν την δυνατότητα κρυπτογράφησης και αποκρυπτογράφησης, υπολογισμού αθροίσματος ελέγχου \emph{(checksum)}, κωδικοποίησης και αποκωδικοποίησης, και συμπίεσης και αποσυμπίεσης μπλοκ δεδομένων σε ένα αρχείο.
Επί του παρόντος, υπάρχει ένας μεγάλος αριθμός προγραμμάτων που είναι σε θέση να κάνει επεξεργασία αυτών των αρχείων.
Κάποια από τα εμπορικά έχουν την δυνατότητα περίπλοκων λειτουργιών όπως είναι το Winhex το οποίο διαθέτει πρόγραμμα επεξεργασίας RAM, παρέχοντας πρόσβαση σε φυσική μνήμη και εικονική μνήμη άλλων διεργασιών.
Όπως διαθέτει επίσης πρόγραμμα επεξεργασίας δίσκων για σκληρούς δίσκους, δισκέτες, CD-ROM & DVD, ZIP, Smart Media, Compact Flash.
Παράλληλα, τα προγράμματα ελεύθερου και ανοιχτού κώδικα όπως και τα εμπορικά διαθέτουν με την σειρά τους πληθώρα λειτουργιών.
Όπως για παράδειγμα scripting με κάποια εξωτερική γλώσσα προγραμματισμού, \emph{inline disassembly} και υποστήριξη ιδιαίτερα μεγάλων αρχείων.

Επίσης τα προγράμματα αυτά αποτελούν απαραίτητο εργαλείο για σενάρια reverse engineering. 
Η πρακτική της αντίστροφης μηχανίκευσης λογισμικού (software reverse engineering) αποτελεί σημαντική πρόκληση ιδιαίτερα στην μελέτη παρωχημένων (legacy) προγραμμάτων δίχως ο πηγαίος κώδικας να είναι διαθέσιμος και στην αντιμετώπιση πιθανών κινδύνων ασφάλειας ενάντια σε ιούς.

Εφαρμόζοντας reverse engineering στα επικείμενα προγράμματα έχει παρατηρηθεί πως είναι αρκετά απαιτητικό και χρονοβόρο. Δυσκολία επίσης συναντάται σε μοτίβα επαναχρησιμοποίησης εκτελέσιμου κώδικα.

\pagebreak
\subsection{Σκοπός της Εργασίας}
% TODO motivation
Ο σκοπός της συγκεκριμένης πτυχιακής εργασίας είναι σε πρώτος μέρος να μελετήσει, αναλύσει τους hex editor που κυκλοφορούν και να υλοποιήσει ένα hex editor για τερματικό με τις βασικές λειτουργίες επηρεασμένες από την ανάλυση.
Σε δεύτερο μέρος να μελετήσει την τεχνική reverse engineering \emph{binary code reuse detection} με τελικό σκοπό την ενσωμάτωσή της στο εν λόγω πρόγραμμα.
Ταυτόχρονα, αυτός ο συντάκτης θα είναι σχεδιασμένος για τις κύριες πλατφόρμες: \emph{MS Windows, Linux, Mac OS}.
Ενσωματώνοντας τεχνικές reverse engineering ένας \emph{hex editor} αποτελεί ένα ισχυρό αλλά και χρήσιμο εργαλείο.

Αναλυτικότερα, οι απαιτήσεις οι οποίες προδιαγράψαμε προκειμένου να υλοποιηθούν από την ανάλυση των hex editor (που ακολουθεί παρακάτω) είναι οι εξής:

\begin{itemize}
    \item \textbf{Μετακίνηση}: Ο χρήστης θα έχει την δυνατότητα να μετακινηθεί σε οποιοδήποτε σημείο του αρχείου θέλει.
    
    \item \textbf{Εύρεση}: Ο χρήστης θα μπορεί να βρει το ή τα σημεία του αρχείου που υπάρχει το μοτίβο με βάση την δοθείσα συμβολοσειρά που θα παρέχει ο ίδιος.
    
    \item \textbf{Αντικαθιστώ}: Ο χρήστης θα μπορεί να αντικαθιστά bytes του αρχείου με bytes που αυτός επιθυμεί.
    
    \item \textbf{Αναγνώριση μορφής}: Το πρόγραμμα θα εντοπίζει την κεφαλίδα του αρχείου. Στην περίπτωση σφάλματος εμφανίζεται το ανάλογο μήνυμα στον χρήστη. 
    
    \item \textbf{Υποστήριξη μεγάλων αρχείων}: Το πρόγραμμα θα υποστηρίζει αρχεία μέχρι 1TB ανοίγοντάς τα σε λογικό χρονικό διάστημα.
    
    \item \textbf{Go To}: Ο χρήστης θα έχει την δυνατότητα να φτάσει σε οποιαδήποτε σημείου του αρχείου μέσα σε ένα πολύ μικρό διάστημα.

    \item \textbf{Code Reuse Detection}: Ο χρήστης θα έχει την δυνατότητα δίνοντας δύο εκτελέσιμα αρχεία να μελετήσει τα σημεία τα οποία ο εκτελέσιμος κώδικας είναι όμοιος ανάμεσα στα δύο αρχεία.

\end{itemize}

Η τεχνική reverse engineering που θα μελετηθεί, θα ανιχνεύει μοτίβα επαναχρησιμοποίησης κώδικα \emph{(binary code reuse detection)}.
Η ανίχνευση των μοτίβων μπορεί να εφαρμοστεί σε σενάρια όπως λογοκλοπή λογισμικού, παραβίαση αδειών λογισμικού ή binary diffing.
Οι επιστημονικές δημοσιεύσεις που θα βασιστεί αυτή η μελέτη είναι τρεις και θα αναφερθούν παρακάτω.

\pagebreak

\subsection{Δυσκολίες και Προκλήσεις}
Η συγγραφή ενός hex editor δεν αποτελεί μια trivial υλοποίηση καθώς προϋποθέτει ακρίβεια και μεθοδική πρακτική για κάθε μια από τις λειτουργίες που την απαρτίζει.
Συγκεκριμένα, οι δυσκολίες - προκλήσεις που βρέθηκα αντιμέτωπος ξεκινώντας από τις βασικές λειτουργίες ήταν η στοίχιση στο τερματικό, τα σινιάλα \emph{signals} για τα διάφορα callbacks όπως dynamic resizing του τερματικού, για την ενεργοποίηση του \emph{raw mode}...

Σε ό,τι αφορά το κομμάτι της μελέτης - υλοποίησης της τεχνικής \emph{binary code reuse detection} βρέθηκα αντιμέτωπος με απαιτητικές αλγοριθμικές υλοποιήσεις. Ως αποτέλεσμα η υλοποίηση των δύο από των τριών επιστημονικών αναφορών είναι ελλιπής και μερική.

\subsection{Δομή της Εργασίας}
Η παρούσα πτυχιακή χωρίζεται σε δύο μέρη.
Το πρώτο μέρος παρουσιάζει, περιγράφει εν συντομία και αναλύει τους hex editor που κυκλοφορούν στην αγορά.
Το δεύτερο μέρος παρουσιάζει, και αναλύει τρία επιστημονικά άρθρα τα οποία αναφέρονται στο BCRD.
Μια σύντομη περιγραφή της δομής της εργασίας ακολουθεί:

Στο 2ο κεφάλαιο, γίνεται μια εισαγωγή για το τι είναι ένας hex editor.
Έπειτα ακολουθεί μια περιγραφή των ήδη υπαρχόντων υλοποιήσεων στο διαδίκτυο καθώς και μια συγκριτική ανάλυση στο τέλος.
Στο ίδιο κεφάλαιο αναλύονται οι τρεις επιστημονικές αναφορές που επιλέχθηκαν και πραγματοποιείται μια συγκριτική ανάλυση ως προς τις μετρικές και τις υποκείμενες υλοποιήσεις τους.

Στο 3ο κεφάλαιο, θα παρουσιαστούν οι υλοποιήσεις τόσο του hex editor που πραγματεύεται η πτυχιακή όσο και μίας από τις τρεις επιστημονικές αναφορές.

Στο 4ο και τελευταίο κεφάλαιο, θα αναφερθούν τα συμπεράσματα της ανάλυσης και των υλοποιήσεων καθώς και μελλοντικές επεκτάσεις που ενδέχεται να πραγματοποιηθούν.
