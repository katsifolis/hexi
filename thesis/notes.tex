\begin{enumerate}
    \item Εισαγωγή
    \begin{itemize}
        \item  Σκοπός Εργασίας
        \item Τι;
        \item Κίνητρα (Motivation)
        \item Γιατί; Πού μπορεί να φανεί χρήσιμο αυτό που φτιάξατε/αναλύσατε;
        \item Μεγαλύτερες προκλήσεις που αντιμετωπίσατε κατά την υλοποίηση συνοπτικά
    \end{itemize}
    \item  State of the art 
    \begin{itemize}
        \item ανάλυση εργαλείων και τεχνολογιών τα οποία είναι ήδη διαθέσιμα για να υλοποιούν όλο αυτό που φτιάξατε ή ένα μεγάλο μέρος. Ανάλυση χαρακτηριστικών και αδυναμιών/ελλείψεών τους.
        \item ανάλυση ανά πρόκληση (τι λύσεις υπάρχουν στη βιβλιογραφία για κάθε πρόκληση; Ανάλυση για κάθε μία λύση και αναφορά στα προβλήματα που παραμένουν άλυτα ή στις αδυναμίες/ελλείψεις της κάθε λύσης)
    \end{itemize}
    \item Προσέγγιση
    \begin{itemize}
        \item Διατύπωση των προκλήσεων αναλυτικά
        \item Αρχιτεκτονική, εργαλεία, ανάλυση λύσεων που επιλέξατε σε σχέση με τις προκλήσεις και λεπτομέρειες. (Μπορεί να καταλήξατε σε μία δική σας λύση για κάθε μία πρόκληση. Αυτό είναι ΟΚ)
    \end{itemize}
    
    \item  Αποτελέσματα
    \begin{itemize}
        \item  Σύγκριση λύσεων για κάθε πρόκληση που βρήκατε από το κεφάλαιο 2 όταν τελικά τις υλοποιήσατε
        \item Εναλλακτικά, παροχή κάποιων μετρήσεων που να δείχνουν πώς τα πήγε η λύση που επιλέξατε. Π.χ. αν η μετρική που έχει νόημα σε μία πρόκληση είναι ο χρόνος, τότε να δείξετε τι χρόνους πετυχαίνει η λύση σας. Δε χρειάζεται να είναι τέλειοι χρόνοι, μόνο να δώσετε τις τιμές και μία εξήγηση για αυτές τις τιμές.
    \end{itemize}
    \item  Συμπεράσματα
    \begin{itemize}
        \item Σύνοψη της δουλειάς σας
        \item Βασικά συμπεράσματα
        \item Μελλοντική δουλειά
    \end{itemize}
\end{enumerate}










 - Detecting Code Clones in Binary Executables: Code region normalization.
