\section{Συμπεράσματα}

Ο πρώτος στόχος της εργασίας ήταν να δημιουργήσει έναν hex editor για το τερματικό, ο οποίος θα διέθετε τουλάχιστον τα βασικά χαρακτηριστικά κοινά για άλλους hex editors και θα έτρεχε στις γνωστές πλατφόρμες Windows, macOS και Linux.
Πραγματοποιήθηκε ανάλυση των τρεχόντων εκδοτών HEX, οι οποίοι καθόρισαν τις βασικές λειτουργίες που θα περιείχε το πρόγραμμα που αναπτύχθηκε.
Στη συνέχεια, σχεδιάστηκε, εφαρμόστηκε και δοκιμάστηκε ένας νέος hex editor , ο οποίος πληρεί τις απαιτήσεις που ορίζονται παραπάνω.
Αυτός ο επεξεργαστής hex προσφέρει μια διεπαφή παρόμοια με τον \emph{kilo} text editor στον οποίο βασίστηκε και παρέχει βασικές λειτουργίες αυτών των hex editor που μελετήθηκαν.

Ο δεύτερος στόχος της εργασίας ήταν να μελετήσει μια τεχνική reverse engineering την \textbf{binary code reuse detection} πάνω στην βιβλιογραφία.
Πραγματοποίηθηκε μια μελέτη πάνω σε τρεις επιστημονικές αναφορές οι οποίες περιέγραφαν την δικιά τους οπτική και υλοποίηση πάνω στην τεχνική αυτή. 

H προσωπική αποκόμιση μετά από την ανάλυση και την υλοποίηση των θεμάτων της πτυχιακής ήταν η γνωριμία με το τερματικό και τις αντίστοιχες βιβλιοθήκες.
Επίσης μια ουσιαστίκη γνωριμία με τον όρο reverse engineering και ειδικότερα με την τεχνική binary code reuse detection.
Τέλος η συγκριτική ανάλυση και η αναζήτηση της βιβλιογραφίας ήταν εξίσου σημαντίκες γνώσεις.

Επιπλέον, η εργασία περιέγραψε τις επεκτεινόμενες λειτουργίες που θα μπορούσαν να συμβούν στο μέλλον.

\pagebreak
\subsection{Μελλοντικές Επεκτάσεις}
Έχοντας επενδύσει αρκετό χρόνο στην υλοποίηση του editor και στην μελέτη τέτοιου είδους προγραμμάτων πιστεύω είμαι σε ένα ικανοποιητικό στάδιο για να απαριθμήσω συγκεκριμένες επεκτάσεις για τον editor της πτυχιακής.

Αυτό το κεφάλαιο θα περιγράψει πιθανές επεκτάσεις του προγράμματος.
Υπάρχουν χρήσιμα χαρακτηριστικά μεταξύ των editor, τα οποία δεν περιλαμβάνονται στα απαραίτητα στοιχεία αυτής της εφαρμογής
\begin{itemize}
    \item \textbf{Πρόσθετη κωδικοποίηση κειμένου}: Υποστήριξη για πολλαπλές κωδικοποιήσεις κειμένου ή εγγραφή σε άλλες κωδικοποιήσεις εκτός από ASCII. Οι κωδικοποιήσεις πολλαπλών byte θα εκμεταλλευτούν την ύπαρξη ενός τρόπου εισαγωγής για τη σύνταξη κειμένου.
    \item \textbf{Τροποποίηση της μνήμης της διαδικασίας ως αρχείο}:
      Δυνατότητα επεξεργασίας της μνήμης διαδικασίας ως κανονικού αρχείου. Αυτή η λειτουργία  θα διαβάζει το `αρχείο` χρησιμοποιώντας το API του εκάστοστε συστήματος για να διαβάσει τη μνήμη.
    \item \textbf{Προσαρμογή της οπτικής εμφάνισης του προγράμματος επεξεργασίας}:
      Προσαρμογή των γραμματοσειρών που χρησιμοποιούνται, καλύτερη επισήμανση με την χρήση χρωμάτων, colorthemes.
    \item \textbf{Ενσωμάτωση του Code Reuse Detection}: Ο χρήστης θα έχει την δυνατότητα δίνοντας δύο εκτελέσιμα αρχεία να μελετήσει τα σημεία τα οποία ο εκτελέσιμος κώδικας είναι όμοιος ανάμεσα στα δύο αρχεία.
	\item \textbf{Τροποποίηση στο configuration κατά την έναρξη}: Ο χρήστης θα έχει την δυνατότητα να παρέχει κάποια flags για να αλλάζει π.χ τον αριθμό των στηλών, την βάση (από δεκαεξαδική σε οκταδική, δεκαδική).
\end{itemize}
