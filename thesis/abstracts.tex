% \addcontentsline{toc}{section}{Περίληψη}
\section*{Περίληψη}
Η συγκεκριμένη πτυχιακή εργασία κλήθηκε από την μια μεριά να παρουσιάσει, να μελετήσει και να συγκρίνει τους διάφορους hex editor που κυκλοφορούν με απώτερο σκοπό να υλοποιήσει ένα νέο hex editor βασισμένο σε λειτουργίες των εν λόγω προγραμμάτων.
Από την άλλη μεριά επισκοπεί να παρέχει μια εισαγωγή στην έννοια του reverse engineering \emph{RE} όπως και να μελετήσει μια τεχνική η οποία ονομάζεται \emph{Binary Code Reuse detection} μέσω τριών επιστημονικών άρθρων.
Η γλώσσα προγραμματισμού που επιλέχθηκε είναι η \emph{c} για την ταχύτητα που προσφέρει αλλά και την ευκολότερη διεπαφή με το τερματικό για το οποίο θα αναπτυχθεί ο editor.
Ορισμένα από τα βασικά χαρακτηριστικά του είναι η τροποποίηση ξεχωριστών μεμονωμένων byte, λειτουργία αντικατάστασης, αναζήτηση διεύθυνσης, αναγνώριση αρχείων (από την κεφαλίδα).


\vspace{1cm}
\noindent\textbf{Λέξεις κλειδιά}: [hex editor, multi-platform, μεγάλα αρχεία, reverse engineering, ανίχνευση επαναχρησιμοποίησης κώδικα]

\pagebreak
% \addcontentsline{toc}{section}{Abstract}
\section*{Abstract}
This dissertation was invited on the one hand to present, study and compare the various hex editors that are circulating with the ultimate goal of implementing a new hex editor based on the functions of these programs.
On the other hand, it intends to give an introduction to the concept of reverse engineering \emph{RE} as well as to study a technique called \emph{Binary Code Reuse detection} utilizing three scientific papers.
The programming language selected is \emph{c} for the speed it provides and easy interface with the terminal for which the editor will be developed.
Some of its key features are modification of individual bytes, replacement function, memory address offset search, file recognition (from the header)

\vspace{1cm}
% Adjust this vertical space as needed
\noindent\textbf{Keywords}: [hex editor, multi-platform, big files, reverse engineering, binary code reuse detection]
